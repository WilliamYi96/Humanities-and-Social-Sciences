\documentclass[UTF8]{ctexart}
\usepackage{amsmath}
\usepackage{graphicx}
\usepackage{float}
\usepackage{subfigure}
\usepackage{xeCJK}
\usepackage{hyperref}
\usepackage{algorithm2e}
\usepackage{amsfonts}
\usepackage{epsfig}
\usepackage{listings}
\usepackage{xcolor}
% 定义可能使用到的颜色

\definecolor{CPPLight}  {HTML} {686868}
\definecolor{CPPSteel}  {HTML} {888888}
\definecolor{CPPDark}   {HTML} {262626}
\definecolor{CPPBlue}   {HTML} {4172A3}
\definecolor{CPPGreen}  {HTML} {487818}
\definecolor{CPPBrown}  {HTML} {A07040}
\definecolor{CPPRed}    {HTML} {AD4D3A}
\definecolor{CPPViolet} {HTML} {7040A0}
\definecolor{CPPGray}  {HTML} {B8B8B8}
\lstset{
    columns=fixed,
    numbers=left,                                        % 在左侧显示行号
    frame=none,                                          % 不显示背景边框
    backgroundcolor=\color[RGB]{245,245,244},            % 设定背景颜色
    keywordstyle=\color[RGB]{40,40,255},                 % 设定关键字颜色
    numberstyle=\footnotesize\color{darkgray},           % 设定行号格式
    commentstyle=\it\color[RGB]{0,96,96},                % 设置代码注释的格式
    stringstyle=\rmfamily\slshape\color[RGB]{128,0,0},   % 设置字符串格式
    showstringspaces=false,                              % 不显示字符串中的空格
    language=c++,                                        % 设置语言
    morekeywords={alignas,continute,friend,register,true,alignof,decltype,goto,
    reinterpret_cast,try,asm,defult,if,return,typedef,auto,delete,inline,short,
    typeid,bool,do,int,signed,typename,break,double,long,sizeof,union,case,
    dynamic_cast,mutable,static,unsigned,catch,else,namespace,static_assert,using,
    char,enum,new,static_cast,virtual,char16_t,char32_t,explict,noexcept,struct,
    void,export,nullptr,switch,volatile,class,extern,operator,template,wchar_t,
    const,false,private,this,while,constexpr,float,protected,thread_local,
    const_cast,for,public,throw,std},
}

\graphicspath{{images/}}
\setCJKmonofont{Microsoft YaHei}

\title{\Huge{中国社会变革新形势下的\\信任缺失与信任重建}}
\author{\Huge{易凯}}
\date{\Huge{2017年2月27日}}

\begin{document}
	\maketitle
	\vspace{35mm}
	\begin{flushright}
	\Large{
  	\textbf{班\ \ \ \ \ 级} \makebox[5em][l]{软件53班}

  	\textbf{学\ \ \ \ \ 号} \makebox[5em][l]{2151601053}

  	\textbf{邮\ \ \ \ \ 箱} \makebox[5em][l]{williamyi96@gmail.com}

  	\textbf{联系电话} \makebox[5em][l]{13772103675}

  	\textbf{个人网站} \makebox[5em][l]{https://williamyi96.github.io}

  	\textbf{提交日期} \makebox[5em][l]{2017年5月8日}
  	}
  	\end{flushright}
  	\newpage
  	\tableofcontents
	\newpage

    \section{摘要}
    近些年来,随着市场经济在中国逐步走向成熟,随着社会体制改革的不断深化,随着互联网+、人工智能时代的到来,社会信任受到了严重的冲击,其不仅包括人与人之间的信任缺失,同时也包括了个体对于政府信任的进一步缺失,社会信任的底线同时也受到了严重的破坏,其在一定程度上已经影响了人际间的正常交往和社会的正常发展。信任缺失问题成为了阻碍新时期改革社会结构健康发展的绊脚石,解决好该问题能够促进社会改革的顺利进行。而信任缺失问题的成因是社会变革新时期信任模式的交汇更替、价值观扭曲、社会经济运行制度规则异化、互联网推动下虚拟化经济等多个方面因素共同导致的,解决的方法建议可以从大力加强社会公共资源建设、大力加强社会道德与中国梦建设、大力加强政府公信力建设、大力加强社会信任应对调控机制建设这四个方面进行。

    \paragraph{关键词}  社会变革\  \ \ \ 新形势\ \ \ \ 信任\ \ \ \  信任缺失\ \ \ \ 信任重建

    \section{论文写作基本说明}
    该论文是西安交通大学本科生易凯于2017年春选修《社会学概论》的期末论文文章,所有内容是在查看大量资料以及课程学习的基础之上对于相关问题的思考与总结。该论文主要探讨的是互联网发展如此迅猛的当代社会信任感缺失的问题,以及如何采取有效的措施进行信任重建。由于本人为软件工程专业学生,对于互联网新时期下的社会选择有一定的认识与了解,因此解决方式的主要立足点于此,另外,由于本人能力以及写作时间有限,文章中不乏有不当或者缺漏的地方,敬请读者批评指正。

    \section{信任的概念及其作用}
    信任这个概念自古有之,不同的时代以及不同的领域都有着不同的认识与定义,在社会科学领域,个人比较欣赏的一种解释是: 信任是一种依赖关系,同时也是一种社会资源。 由信任构建起来的依赖关系如卢曼所说,是为了简化人与人之间的合作关系而发展起来的,值得信任的个人或者团体y意味着他们寻求一定的实践政策、道德守则以及先前的承诺。同时,信任也是一种社会资源,是人际关系的精神纽带,是进一步发展密切关系的基础;是简化复杂性的重要机制,是经济交易和合作的必要条件。这种社会资源能够引导一种价值取向,促进社会成员之间交往合作关系的进一步产生。

    信任可以认为是文明社会发展的最重要的综合动力机制之一。一个社会信任健全的社会,不但可以增加公民的社会安全感、社会责任感、社会公平感,减少社会交往的复杂性,提高社会的公信力,而且社会信任还构成了一个社会经济繁荣和稳定的心理基础。正是由于社会群体之间信任的存在,社会才能够成为一个相互联系而彼此依赖的整体。

    \section{当代社会信任问题的现状}
    近些年来,随着市场经济在中国逐步走向成熟,随着社会体制改革的不断深化,随着互联网+、人工智能时代的到来,社会信任受到了严重的冲击,其不仅包括人与人之间的信任缺失,同时也包括了个体对于政府信任的进一步缺失。

    在经济、文化日益繁荣的当下,社会信任的底线同时也受到了严重的破坏,其在一定程度上已经影响了人际间的正常交往和社会的正常发展。

    当下,放之小处,部分公民诚信缺失、政务诚信缺失、商务诚信缺失、公共服务领域诚信缺失、部分干部领导诚信缺失,电子商务平台部分商家诚信缺失,可以说以及渗透到了当代生活的各个方面。目前社会公信力下降导致了更为严重的社会信任危机,政府表态缺乏说服力,专家解释缺乏可信度被拍砖,媒体报道严重失真甚至扭曲。政府、专家、媒体等曾经权威的代表如今普遍受到质疑。我们从近年来不断发生的社会冲突中可以看到,这种问题越演越烈。曾经的毒奶粉受到争议,曾经的老人跌倒扶不扶的问题引发热议,在各行各业,似乎很少能够见到能够让全社会都公认充分信任的行业或者职业了。

    因此,研究并解决社会信任缺失问题具有着迫切性,下面就来分析一下信任缺失的根本原因。

    \section{信任缺失的根本原因}
    新中国60多年的顽强奋进,改革开放30多年乘风破浪,使中国社会走到了一个历史的转折点。目前中国社会变革的进程正不断向前推进,李毅认为,当前社会变革的三大当务之急是:应尽快把生产模式转变为以自主创新为主;改革城乡二元化、地区隔离的户籍制度;反腐倡廉,颁行《官员财产申报法》。

    而我也想主要依据此三点来展开进行新形势之下信任缺失问题的本质分析:

    \subsection{社会变革新时期信任模式的交汇更替衍生出了信任缺失}
    社会学家将社会信任分为特殊信任和普遍信任、熟人信任和陌生人信任、人格信任和制度信任等等,这些分类表述不尽相同,但就信任的基本模式来看,实质上只有两大类:传统信任模式和现代信任模式,核心就是道德信任和法制信任。传统社会中,人际交往大体是在血缘和地缘基础之上展开的,信任的保障机制建立在道德关系加上个人特质的基础之上,其突出特点就是熟人信任与道德信任。但是在现代社会中,随着工业化、城市化、市场化的进程和经济文化的发展,过去的地缘血缘范围被打破,人际交往领域大幅拓展,交往对象频繁更替,良好的道德人品声望不再是最邮箱的信任保障因素,而必须依靠制度和规则进行维系,依靠法治进行支撑,从而可以说是依靠制度信任、法治信任。

    当前,我国正处于社会变革新时期,社会信任随着社会转型而转型。在这个过程中,传统的信任模式已经不能够满足现代社会经济发展的需要,而现代信任模式还在生成之中。出于这种新旧交替的阶段,人们总是习惯从传统信任的角度观察判断社会变革新时期的各种现实问题,可想而知,难以找到满意的答案,同样也不可能在生成的社会新人中找到能够说服自己和他人的答案。在这种矛盾与冲突中,催生了信任缺失。

    \subsection{社会变革新时期价值观扭曲导致了信任缺失}
    改革开放使我国告别传统的计划经济进入现代市场经济社会。二者相比,前者政治、经济、社会、文化等高度统一,思想意识形态高度一致,利益及其获取渠道高度单一,从摇篮到坟墓国家全包,因而没有太多的利益追求,价值取向表现为舍我、重义、诚信为天;后者经济成分多元,思想文化多元,利益主体多元,利益格局多元,价值取向表现为自我、趋利、金钱至上。

    社会变革的新时期,市场化的进一步全面渗透,金钱与物质满足成为追逐的中心,道德诚信的观念在时代潮流变革之下急剧淡化。人们在追求价值目标时,致力于采用最低廉的成本、最“有效”的方法去争取最大的受益。为了取得最大化的利益,人们可能不再遵循传统的道德规范,不讲诚信,不择手段,不去考虑对他人、对社会以及对自己长远利益的危害,这在博弈论中也就是“从一开始就输掉的博弈”。与此同时,这种带有明显转型期社会特点的功利取向,动摇了人的道德底线,扭曲了传统的价值观念,驱使人们无视道德廉耻,无视制度规则,见利忘义唯利是图。其结果必然是破坏社会信任的基础,信任缺失也就难以避免。

    同时也正是由于社会变革新时期下各阶层、各群体之间的利益关系、效率和公平的关系还处于调整变化之中,因此社会变革新时期的价值扭曲导致的信任缺失才如此显著。

    \subsection{社会变革新时期社会经济运行制度规则异化加剧了信任缺失}
    市场经济条件下,国家和社会所制定的制度规则以及对制度规则的强力维护和对违反行为的严厉处罚,是现代社会经济运行的必备条件,同样也是现代社会信任的重要保证。然而在现实生活中,社会经济运行所依赖的这些制度规则受到了无序竞争的挑战而出现严重的异化。

    一方面,潜规则盛行与社会经济生活,显规则被束之高阁。尤其是在社会公共资源配置,重大项目的招投标以及其他各种市场竞争中,供需双方通过投机经营、暗箱操作、利益分成等潜规则的方式和手段来实现自己利益目标并使其最大化。另外一方面,许多的“昏规则”也加剧了社会信任的确实,例如当前中国有很多典型的刑事案件的处理,或者证据不足,或者真相不明,或者是判决表述推断片面主管,将助人者陷于道德和法律的风险和困境之中。很多时候,这些昏规则让好心人的心逐渐变冷,偏转了道德的罗盘,也自然异化了人们的价值判断,“潜规则”和“昏规则”的双重打击之下,自然导致了社会信任缺失的进一步加剧。

    \subsection{社会变革新时期互联网推动下虚拟化经济加剧了信任缺失}
    二十一世纪是属于互联网信息化的世纪,特别是进入到2010年之后,智能手机以常人难以想象的进展在向前发展,在互联网新技术的浪潮下涌现出的虚拟现实技术,增强现实技术等最新科技,给用户带来了前所未有的新奇体验。同时,电子商务进一步发展,当今社会可以说电子商务已经渗透到了生活的方方面面,货币似乎正在逐渐被虚拟化,交往对象以及对交往对象的信任也似乎在逐渐被虚拟化,作为未来社会骨干的青少面门往往更倾向于互联网上哪些与他们志趣相投的群体成员,而普遍缺乏对于当下许多所谓“专家”的信任。

    另外一方面,社会市场经济的虚拟化也让信息真假的甄别成为了一件格外棘手的事情,百度的“莆田系”医院广告推广则是一个很好的范例,当面对着纷繁错杂、真真假假的各种信息,更多的人就选择了对于大部分信息采取怀疑的态度,同时也将这种态度来指导社会生活,自然加剧了社会信任的进一步淡漠与缺失。

    \section{新时期进行信任重建的参考形式}
    \subsection{大力加强社会公共资源建设}
    当前社会个体之间相对独立,同在相邻的地理位置而彼此互不熟悉的现状成为一种常态,每个人的社会角色更加趋向于个性化,而缺少了集体主义精神,而高度的利己化自然导致了社会信任的缺失。通过加大社会公共资源建设,例如如今正在推广的共享单车的方式,加强社会成员之间的公共性,这是一种联系社会成员的重要方式,同时也将促进社会关系的优化,解决现有的许多社会矛盾。而共享公共资源本身则建立在交互与继承的基础之上的,因此有利于培养人与人之间的信任,同时有利于培养社会团队之间的信任。

    \subsection{大力加强社会道德与中国梦建设}
    社会道德建设是整个社会群体价值导向的核心,一方面,要加强全民道德建设。只有深化社会主义核心价值观的根本内容,不断地用社会主义核心价值体系引领社会思潮,凝聚社会共识。加强社会公德、职业公德、家庭美德、个人品德教育,弘扬中华传统美德,弘扬时代新风,才能加强全道德建设成果。同时要发挥道德楷模先进典型的示范作用,引导公民增强道德判断力和道德荣誉感,摒弃观念的功利化、行为的短期化,在全社会形成讲正气、比奉献的良好风尚。另外一方面,积极推动社会道德法制化。这也是解决当前新形势之下社会道德困境、提升社会信任水平的重要思路。我们需要明确法律概念,让全民知法懂法,将道德化的法制意识深深植根于普通公民心中。

    此外,中国梦作为当前的政治发展主流,其内涵不仅仅是国家的梦,更是每一个公民自己的梦。加强中国梦的建设,搭配上社会主义核心价值观,有利于从思想信念上塑造诚信的理念,不断地提升集体荣誉感。同时在同一个目标的推动之下,有利于促进民族的团结,不断地推动社会变革的进程向着更好更快发展。

    \subsection{大力加强政府公信力建设}
    政府的公信力作为社会信任的基础,当前受到了不小的动摇。只要打造诚信政府,提升政府的公信力,塑造良好的政府形象,才能在源头上提升社会信任的水平,推动新时期之下的社会信任重建。

    具体而言,实施的途径主要有以下几个方面:

    一是要建立诚信法规制度体系。其主要内容包括政务诚信、商务诚信、社会诚信、司法诚信等多个方面,在操作层面上进行严密的规则设计,用以规范政务、商务、社会和司法行为。并在此基础之上,建立诚信档案和覆盖全国的征信系统,实现信用信息的互联互通,在社会上营造良好的诚信环境。

    二是要认真履行政府监管职责,重点是与老百姓相关的民生问题,例如食品安全、药品安全、社会医疗保障等,重点投入,重点监管,同时要将这种监管进行体制化,常态化。同时在监管过程中加强市民意见的反馈与收集。

    三是要建立相应的诚信激励机制。当前市场化经济是社会变革的很重要的一个成分,我们要对诚实守信者大力表彰和宣传,对失信的行为降低容忍,严厉处罚,以期提高风险成本。从而扭转当先扭曲价值观中唯利是图的成分,从长远的角度去提升社会信任水平。

    \subsection{大力加强社会信任应对调控机制建设}
    社会信任应对调控机制对于社会变革新时期的发展起到了突出作用,我们既要从政治生态、社会生态、经济生态的角度出发,将社会信任纳入社会建设管理的总体框架之内,明确专门机构负责,专门处理社会信任问题的应对与调控和相关理论和实践研究。另外,要借鉴相关的优秀做法,开放关于社会信任指数的民意调查窗口,定期发布相关的数据信息,在一种直观、透明、量化的基础之上引导全社会更加理性的判断、科学决策并且进行有效应对。此外,极为重要的一个方面是,要建立社会信任问题响应与调节机制。面对社会普遍关注和质疑的问题,要及时收集舆情,积极响应、调节和处理,加强舆论的引导,认真回复相关回应,将防止信任问题进行常态化,从源头上来处理社会信任问题。

    \section{总结与收获反思}
    信任缺失在中国社会变革的新形势之下成为了一个亟待解决的问题,某种程度上能够在新时期进行好社会重建的事业决定了中国未来的社会发展结构走向。信任缺失问题的成因是社会变革新时期信任模式的交汇更替、价值观扭曲、社会经济运行制度规则异化、互联网推动下虚拟化经济等多个方面因素共同导致的,解决的方法建议可以从大力加强社会公共资源建设、大力加强社会道德与中国梦建设、大力加强政府公信力建设、大力加强社会信任应对调控机制建设这四个方面进行。通过加强这四个方面的建设,将中国梦放在每个人的心中,才能够去拥抱更加美好的社会未来!

    半个学期的社会学的学习至此拉下了帷幕,在这个过程中,查阅了很多的社会学及其交叉学科的很多资料,同时也对于许多社会问题进行了透视思考,确实收益甚广。作为一名软件工程专业的学习,对于社会学问题的思索有利于对这个世界有更为清晰的认识,以期来指引未来的学习生活。追梦的路上我不会停下脚步,相信未来会更美!

    \section{致谢}
    Give my sincere thanks to my sociology teachers for his excellent teaching skills and serious altitude for our homework. Give my sincere thanks to some students who have helped me and who have inspired me when discussing with them. Give my sincere thanks to myself because I've overcome all difficulties and have successfully finished my sociology paper. Give my sincere thanks to those pioneers who have devoted themselves to writing immortal books. Give my sincere thanks to all the people sharing their ideas and harvests without pay in QA communities.

    衷心感谢所有社会学老师们出色的教学风范以及严谨的治学态度,衷心感谢那些在讨论中帮助和启发我的同学们,衷心感谢克服了种种困难最终完成了社会学论文的自己,衷心感谢那些写了不朽著作为后人指明道路的先驱们,衷心感谢那些在问答社区无私奉献自己智慧成果的所有同仁!
    
    \section{参考文献}
    [1] C.Wright Mills. The Sociological Imagination. 中国传媒大学出版社. 2016年3月.

    [2] 理查德·谢弗. 社会学与生活.第11版. 世界图书出版公司. 2014年6月.

    [3] Barbara Misztal, Trust in Modern Societies: The Search for the Bases of Social Order, Polity Press, ISBN 0-7456-1634-8.

    [4] Riki Robbins, Betrayed!: How You Can Restore Sexual Trust and Rebuild Your Life, Adams Media Corporation, ISBN 1-55850-848-1.

    [5] Ed Gerck, in Trust Points, Digital Certificates: Applied Internet Security by J. Feghhi, J. Feghhi and P. Williams, Addison-Wesley, ISBN 0-20-130980-7, 1998.

    [6] 高星. 转型期社会信任缺失问题研究. 湖北社会科学. 2013年第4期.

    [7] 高松雪. 转型时期我国政府的信任缺失及其对策性分析. 东北师范大学. 硕士学位论文. 2008年5月.

    [8] 刘越. 中国现代社会的信任缺失与重构. 河南师范大学. 硕士学位论文. 2013年4月.

    [9] 董建军. 信任缺失:传统文化视角的解析. 长春工业大学学报.社会科学版. 2010年7月. 第4期. 第22卷.

    [10] 宾建光. 试论我国当前社会转型中的信任缺失与信任重建. 广西师范大学. 硕士学位论文. 2001年4月.

    [11] 温艳丽. 我国社会转型期的信任缺失与构建. 北京邮电大学. 硕士学位论文. 2010年3月.

    [12] 康德. 三大批判合集(上). 人民出版社. 2010年1月.

    [13] 臧豪杰. 我国社会信任的现状及其重建探论——社群主义视角[J]. 理论导刊, 2011, (8).

    [14] 郑也夫, 彭泗清. 中国社会中的信任[M]. 北京: 中国城市出版社, 2003.

    [15] 郑永年, 黄彦杰. 中国的社会信任危机[J]. 文化纵横, 2011, (4).
\end{document} 